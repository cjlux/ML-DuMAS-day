\documentclass[10pt,serif,mathserif,compress,hyperref={colorlinks}]{beamer}
\mode<presentation>
\usepackage{times} % pour avoir une belle police avec beamer
\usepackage{pgf}
\usepackage{pgfpages}
\usepackage[T1]{fontenc}
\usepackage[utf8]{inputenc}
%\usepackage[francais]{babel}
\usepackage{lmodern}
\usepackage{lastpage}
\usepackage{comment}
\usepackage{geometry}
\usepackage[most]{tcolorbox}
\tcbuselibrary{skins}
\usepackage{beamerthemesplit}
\usepackage{amsmath, amsfonts, epsfig, xspace}
\usepackage{pstricks,pst-node}
\usepackage{multimedia}
\usepackage{pifont}   % zapf dingbats
\usepackage{marvosym} % MarVoSym dingbats
\usepackage{wasysym}

\usepackage{graphicx}% for including figures
\usepackage{tikz}
\usepackage[tikz]{bclogo}
\usetikzlibrary{positioning,decorations.pathreplacing,arrows}
\usepackage{tikzsymbols}
\setlength{\parindent}{0pt}

\usepackage{minted}

\input{colors}
\input{commands}

\usetheme{jlcKeynote}
\useoutertheme[subsection=false]{miniframes}
\setbeamercolor{background canvas}{bg=gray!50!white}
\setbeamercolor{structure}{bg=white, fg=gray}
\setbeamertemplate{itemize item}{\small\gray{$\CIRCLE$}}
\setbeamertemplate{itemize subitem}{\tiny\gray{$\CIRCLE$}}
\settowidth{\leftmargini}{\usebeamertemplate{itemize item}}
\addtolength{\leftmargini}{\labelsep}

\hypersetup{linkcolor=Yellow}
\hypersetup{citecolor=DeepPink4}
\hypersetup{urlcolor=DarkBlue}
\hypersetup{anchorcolor=Magenta}

\title[\hspace*{.75\linewidth}\insertframenumber/\inserttotalframenumber]
      {\fontsize{17}{17}\selectfont{\textbf{Machine learning with Python:\\Create, install \& use a \\Python Virtual Environment (PVE)}}\\[6mm]
      \fontsize{12}{12}\selectfont{\textbf{DuMAS department day -- 2023/09/22}}
}
      
\subtitle{}
      
\author[{\tiny{JLC -- Sept23 -- v1.1 }}

\hspace*{.75\linewidth}]
%{\includegraphics[height=2.5cm]{images/logo-am-couleur-72dpi_alpha.jpg}\\[5mm]
{\fontsize{9}{9}\selectfont{\hspace*{-5mm}Jean-Luc.Charles@mailo.com}}

\institute{}

\date{}

\titlegraphic{\vspace*{-1.6cm}\includegraphics[height=3.cm]{images/robot.png}\\
  \href{https://creativecommons.org/licenses/by-sa/4.0/}
       {\includegraphics[height=5mm]{images/CC-BY-SA.jpeg}}     
}

\logo{}

\tcbset{enhanced, boxrule=0.2pt, sharp corners, drop lifted shadow,
    width=1.0\textwidth, left=5pt, left skip=-20pt,
    colback=Chocolate!25!white,colframe=Chocolate!75!black}

\renewcommand\ttdefault{lmtt}

\begin{comment}
v1.2 -- JLC :
>>> Modification des tcolorbox pour les avoir :
- plus larges (width=...)
- mieux centrées dans la page  (left skip=...)
- avec moins de marge à gauche (left=...)
\end{comment}

\begin{document}

\frame[plain]{\titlepage}

\setbeamercolor{structure}{fg=gray!50!white}

%===============================================================================
\begin{frame}
  
  \begin {bclogo}[noborder=true, couleur=gray!50, couleurBarre=Chocolate, logo=\bctrombone, marge=0, margeG=-0.5]
      {\ Programming {\em Machine Learning} (ML) in Python3}
      \medskip
      \begin{itemize}
      \item \bfdarkchoco{miniconda3} allows you to install a {\bf dedicated} \bfdarkchoco{Python Virtual Environment} (PVE) on your laptop
        GNU/Linux, macOS or Windows.\medskip
      \item IDE\footnote{Integrated Development Environment} interesting for ML in Python:
        \begin{itemize}
        \item \bfdarkchoco{jupyter notebook}: for creating Python {\em notebooks} $\leadsto$ files \fileBF{*.ipynb}\\ for ML, data processing, reports... 
          Used in most tutorials on the internet.\medskip
        \item \bfdarkchoco{idlex}: the simplest IDE for creating/running \fileBF{*.py} files\\
          (a "Python interpreter" window and a "program editor" window)\medskip
        \item \bfdarkchoco{VSCode}, {\em a.k.a Visual Studio code} from Microsoft: multi-language, very powerful, requires some work (time) to get started,
          especially to make it work with PVE...\medskip
        \item \bfdarkchoco{pycharme}, \bfdarkchoco{pyzo, }\bfdarkchoco{spyder} and many others \href{https://wiki.python.org/moin/IntegratedDevelopmentEnvironments}{here} \ldots
        \end{itemize}
      \end{itemize}
      \medskip
  \end{bclogo}
  
\end{frame}
%===============================================================================

\section{Miniconda3}

%===============================================================================
\begin{frame}{The Python installer : {\bf Miniconda3}}
  
  \begin{tcolorbox}[title=Installation of Miniconda3]

    \begin{itemize}
      
    \item Download the latest version of \fileBF{Miniconda3} for your OS at \href{https://docs.conda.io/en/latest/miniconda.html}{\DarkBlue{docs.conda.io/en/latest/miniconda.html}}.\\
      
     \item Start the installation of \fileBF{Miniconda3}... \textbf{note the path for the installation folder \fileBF{miniconda3}} $\leadsto$ this will be used useful later\ldots\\[1mm]
       \DarkGray{\footnotesize [Linux, in a terminal type: \\[-1.5mm]\hspace*{1.5cm}\code{bash ...some\_where.../miniconda3-latest-Linux-x86\_64.sh}]}
    \end{itemize}
    
  \end{tcolorbox}

  \vfill
\hspace*{-20pt}\textbf{Warning} : the path of the installation folder \fileBF{miniconda3} must not contain any {\bf space} or any {\bf accented characters}!
  
  \begin{minipage}{.15\linewidth}
    \hspace*{-20pt}\includegraphics[width=1.\linewidth]{images/dialog-warning-2.png} 
  \end{minipage}%
  \begin{minipage}{.85\linewidth}
    
    \medskip\fontsize{8}{8}\selectfont{
    Windows :\\
    \file{C:\bsh Miconda3} ou \file{C:\bsh Users\bsh Marie\bsh miniconda3} $\leadsto$ \DarkGreen{OK}\\[1mm]
      \file{C:$\backslash$Yoann$\backslash$Mes\,install$\backslash$miniconda3} $\leadsto$ \DarkRed{not OK} (space)\\[1mm]
      \file{C:$\backslash$Users$\backslash$Léon$\backslash$miniconda3} $\leadsto$ \DarkRed{not OK} (accentuated e)

    \medskip      
    MacOSX \& GNU/Linux :\\
      \file{/home/moi/miconda3} ou \file{/Users/moi/opt/miniconda3} $\leadsto$ \DarkGreen{OK}\\[1mm]
      \file{/home/moi/Mes\,install/miniconda3} $\leadsto$ \DarkRed{not OK} (space)\\[1mm]
      \file{/Users/Léon/miniconda3} $\leadsto$ \DarkRed{not OK} (accentuated e)
    }
  \end{minipage}
\end{frame}
%===============================================================================

\subsection{post_install}

%===============================================================================
\begin{frame}

  \begin{tcolorbox}[title=\textbf{miniconda3} post-Installation]
    
    In the terminal, or the "Anaconda prompt" window:

    \begin{itemize}
    \item to disable the automatic activation of the \textbf{base} default PVE:\\
      \codeBF{conda config --set auto\_activate\_base false}
    \item to get information on the \fileBF{Miniconda3} installation:\\
      \codeBF{conda info}
    \end{itemize}
  \end{tcolorbox}

\end{frame}
%===============================================================================

\section{PVE creation \& activation}

%===============================================================================
\begin{frame}{How to create a PVE (Python Virtual Environment)}

  \begin{tcolorbox}[title={\bf PVE creation}]
    In a \textbf{NEW terminal} (macOS, Linux) or an "Anaconda prompt" window (Windows),
    create the \DarkRed{\boldtt{dumas1}} PVE:\\
    \hspace*{10pt}\codeBF{conda create -n dumas1 python=3.8 -y}\vspace*{-1mm}
  \end{tcolorbox}
  
  \begin{tcolorbox}[title={\bf PVE activation}]
    Once the \DarkRed{\boldtt{dumas1}} PVE is created, you must {\bf activate} it to use it:
    
    \begin{itemize}
    \item in the terminal, or the "Anaconda prompt" window type:\\
      \hspace*{10pt}\codeBF{conda activate dumas1}\medskip
    \item the {\em prompt} is now prefixed with \DarkGreen{\boldtt{(dumas1)}}:\\[-1mm]
      \begin{description}
      \item{\file{Windows:\ \ \ \ }} \boldtt{\DG{(dumas1) C:\bsh Users\bsh me>}}
      \item{\file{macOs:\ \ \ \ \ \ }} \boldtt{\DG{(dumas1) /Users/me>}}
      \item{\file{GNU Linux:\ \ }} \boldtt{\DG{(dumas1) user@home \$ }}
      \end{description}
      \end{itemize}
  \end{tcolorbox}
  
\end{frame}
%===============================================================================

\section{PVE populating}

\subsection{3 ways}

%===============================================================================
\begin{frame}{Installation of Python modules}
  
  \begin {bclogo}[noborder=true, couleur=gray!50, couleurBarre=Chocolate, logo=\bctrombone, marge=0, margeG=-.8]
    {Different methods to load modules into the \DarkRed{\boldtt{dumas1}} PVE}\medskip
    
     \begin{itemize}
     \item With an ASCII-YAML file (\file{*.yml}) listing the modules to install and
       the command \codeBF{conda env update -n dumas1 --file <file.yml>}:\\[1mm]
       $\leadsto$ the most efficient for a PVE created with \codeBF{conda}\medskip
      
     \item With an ASCII file (\file{*.txt}) listing the modules to install and
       the command \codeBF{pip install -r <file.txt>}:\\
       $\leadsto$ the most used on internet tutorials... \\
       $\leadsto$ but can lead to incompatibilities between \codeBF{conda} and \codeBF{pip}\medskip
      
     \item By hand with \codeBF{conda install ...} or \codeBF{pip install ...}:\\
       $\leadsto$ the most painful!
     \end{itemize}    
  \end{bclogo}

\end{frame}
%===============================================================================


%===============================================================================
\begin{frame}[fragile]
\frametitle{Installation of Python modules}

\hspace*{-5mm}Examples of files to install some Python modules in the \DarkRed{\boldtt{dumas1}} PVE:\\[-5mm]
  \hspace*{5mm}\begin{minipage}[t]{.35\linewidth}
    \begin {bclogo}[noborder=true, couleur=gray!50, couleurBarre=Chocolate, logo=\bctrombone, marge=0, margeG=-.8]
    {\small YAML format for \codeBF{conda}}
    \begin{minted}[frame=single, fontsize=\footnotesize]{yaml}
name: dumas1
channels:
  - defaults
dependencies:
  - python=3.8
  - tensorflow==2.8.*
  - pandas
  - matplotlib
  - opencv
  - jupyter
  - notebook
  - scikit-learn
  - seaborn
  - pip
\end{minted}
    \end{bclogo}
  \end{minipage}
\hspace*{25mm}\begin{minipage}[t]{.3\linewidth}
    \begin {bclogo}[noborder=true, couleur=gray!50, couleurBarre=Chocolate, logo=\bctrombone, marge=0, margeG=-.8]
    {\small TXT format for \codeBF{pip}}
\begin{minted}[frame=single, fontsize=\footnotesize]{text}
tensorflow==2.8.*
pandas
matplotlib
opencv
jupyter
notebook
scikit-learn
seaborn
\end{minted}
    \end{bclogo}
    \end{minipage}  

\end{frame}
%===============================================================================

\subsection{with conda and a *.yml file}

%===============================================================================
\begin{frame}{Installation of Python modules}

  \begin{tcolorbox}[title={\bf Preferred method}: YAML file + command \textbf{conda}]
    
    \begin{itemize}
    \item The \codeBF{--file} option of the \codeBF{conda env update} command takes the name of an ASCII file
      in YAML format containing the list of Python modules to install.
     \item It is imperative to designate the PVE concerned with the option: \\\codeBF{-n <PVE\_name>}
     \end{itemize}
    
\end{tcolorbox}

  \begin {bclogo}[noborder=true, couleur=gray!50, couleurBarre=Chocolate, logo=\bctrombone, marge=0, margeG=-.8]
    {Populate the \DarkRed{\boldtt{dumas1}} PVE using the file \file{dumas1.yml}}\smallskip
    {\small In a terminal, or "Anaconda prompt" window:\\[-3mm]
      \begin{itemize}
      \item go to the folder holding the YAML file with the \codeBF{cd} command:\\
       \codeBF{cd <path\_of\_the\_folder\_containing\_the\_file\_YAML>}
      \item then install the Python modules in the \DarkRed{\boldtt{dumas1}} PVE:\\
       \code{conda env update -n dumas1 --file dumas1.yml}
      \end{itemize}
    }
  \end{bclogo}
  
  
\end{frame}
%===============================================================================

\section{Launch Jupyter notebook}

%===============================================================================
\begin{frame}

  \begin{tcolorbox}[title={\bf Windows}: launch {\bf jupyter notebook}]
 
    \begin{itemize}
    \item In an "Anaconda prompt" window, with the \DarkRed{\boldtt{dumas1}} PVE {\bf activated}, type:\\
      \codeBF{jupyter notebook}
    \item Access folders on a disk partition other than \file{C:\bsh} (e.g. \file{D:\bsh})\\
      \codeBF{jupyter notebook D:\bsh}\\
      \codeBF{jupyter notebook D:\textbackslash{folder1}\texttt{\char`\\}folder2}\\
    \item Access folders on a USB key mounted on example \file{E:\bsh}\\
      \codeBF{jupyter notebook E:\bsh}
    \end{itemize}
   
  \end{tcolorbox}

  \begin{tcolorbox}[title={\bf macOS \& GNU/Linux}: launch {\bf jupyter notebook}]
      
    \begin{itemize}
    \item In a terminal with the \DarkRed{\boldtt{dumas1}} PVE {\bf activated}, type:\\
      \codeBF{jupyter notebook}
    \item Access a folder anywhere:\\
      \codeBF{jupyter notebook /home/users/me/folder1/folder2}
    \end{itemize}

   \end{tcolorbox}  
\end{frame}
%===============================================================================

\section{Launch idlex}

\subsection{Windows}

\end{document}









\subsection{with pip and a requirement file}

%===============================================================================
\begin{frame}{Installation des modules Python}
  
  \begin{tcolorbox}[title=Fichier ASCII + commande \textbf{pip}]
    \begin{itemize}
    \item L'option \codeBF{-r} de la commande \codeBF{pip} prend le nom d'un fichier ASCII contenant la liste des modules à installer.
    \item Il faut impérativement désigner l'PVE concerné avec l'option
      
      \codeBF{-n <PVE\_name>} 
    \end{itemize}
  \end{tcolorbox}

  \begin {bclogo}[noborder=true, couleur=gray!50, couleurBarre=Chocolate, logo=\bctrombone, marge=0, margeG=-.8]
    {Installer l'PVE bam avec un fichier \file{requirements.txt}}\smallskip
    Dans un terminal (fenêtre "Anaconda prompt") avec l'PVE {\bf bam activé}, taper:\\
    \Chocolate{\boldtt{pip install -r <chemin\_du\_fichier\_requirements.txt}}\\
  \end{bclogo}
  
\end{frame}
%===============================================================================


\subsection{by hand}

%===============================================================================
\begin{frame}{Installation of Python module}
  
  \begin{tcolorbox}[title={\bf for information only }: By hand...]
      \begin{itemize}
      \item With \codeBF{conda}:\\
        \codeBF{conda install module1 modle2 ...}\\
        \codeBF{conda install module1==x.y.z ...} {\small (force version)}\\[1mm]
      \item With \codeBF{pip}:\\
        \codeBF{pip install module1 modle2 ...}\\
        \codeBF{pip install module1==x.y.z ...} {\small (force version)}
      \end{itemize}
  \end{tcolorbox}

  \begin {bclogo}[noborder=true, couleur=gray!50, couleurBarre=Chocolate, logo=\bctrombone, marge=0, margeG=-.8]
    {Install bam PVE modules manually}
    {\small \DarkRed{$\leadsto$ FYI, not recommended: better to use "YAML file + \codeBF{conda}..."}
       
     In a terminal, or an "Anaconda prompt" window, with the bam PVE {\bf activated}, type successively:\\
     \Chocolate{\boldtt{conda install numpy scipy sympy pandas matplotlib imageio opencv}}\\
     \Chocolate{\boldtt{conda install jupyter notebook jupyterlab}}\\
     \Chocolate{\boldtt{pip install idlex}}
     }
  \end{bclogo}
  
\end{frame}
%===============================================================================

%===============================================================================
\begin{frame}

\begin{tcolorbox}[title=\textbf{Windows}: Shortcut to launch \textbf{Idlex}]

    {\small
    \file{<MC>} designating the path to the folder \file{miniconda3} on your computer:
     \begin{itemize}
     \item In the folder \fileBF{<MC>\bsh envs\bsh dumas1\bsh Scripts}: <<\;right-click\;>> on the file \fileBF{idlex.py} to create a <<\; desktop shortcut\;>>.
     \item In the desktop: <<\;right-click\;>> on the \fileBF{idlex.py} shortcut to modify its <<\;properties\;>> in the <<\;Shortcut\; >> tab:\smallskip
     \begin{itemize}
         \item field <<\;target\;>>: put {\bf on a single line}:\\
           \hspace*{-10mm}\code{\fontsize{6}{6}\selectfont{<MC>\bsh condabin\bsh conda.bat activate dumas1 \& <MC>\bsh envs\bsh dumas1\bsh python.exe <MC>\bsh envs\bsh dumas1\bsh Scripts\bsh idlex.py}}\\
           \hspace*{-8mm}\DarkGreen{\footnotesize $\leadsto$ replace <MC> with the path of the folder {\em miniconda3} on your computer.}
           \smallskip
         \item field <<\;Start in\;>>: put the path of a folder in your user tree, or the path of your user folder.
           \smallskip
         \item field <<\;Icon\;>>: click on <<\;Change icon $\leadsto$ Browse\;>> and install the icon \file{<MC>\bsh Lib\bsh idlelib\ bsh Icons\bsh idle.ico}
     \end{itemize}
   \end{itemize}
   }
  \end{tcolorbox}
 
   \begin{tcolorbox}[title={\bf macOS \& GNU/Linux}: launch \textbf{Idlex}]
   In a terminal with the \DarkRed{\boldtt{dumas1}} PVE {\bf activated}, type: \codeBF{idlex}
   \end{tcolorbox}
   
\end{frame}
%===============================================================================
