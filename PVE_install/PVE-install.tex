\documentclass[10pt,serif,mathserif,compress,hyperref={colorlinks}]{beamer}
\mode<presentation>
\usepackage{times} % pour avoir une belle police avec beamer
\usepackage{pgf}
\usepackage{pgfpages}
\usepackage[T1]{fontenc}
\usepackage[utf8]{inputenc}

\usepackage{lmodern}
\usepackage{lastpage}
\usepackage{comment}
\usepackage{geometry}
\usepackage[most]{tcolorbox}
\tcbuselibrary{skins}
\usepackage{beamerthemesplit}
\usepackage{amsmath, amsfonts, epsfig, xspace}
\usepackage{pstricks,pst-node}
\usepackage{multimedia}
\usepackage{pifont}   % zapf dingbats
\usepackage{marvosym} % MarVoSym dingbats
\usepackage{wasysym}

\usepackage{graphicx}% for including figures
\usepackage{tikz}
\usepackage[tikz]{bclogo}
\usetikzlibrary{positioning,decorations.pathreplacing,arrows}
\usepackage{tikzsymbols}
\setlength{\parindent}{0pt}

\input{colors}
\input{commands}
\usepackage{minted}

% Beamer theme
\renewcommand\sfdefault{phv}
\renewcommand\familydefault{\sfdefault}
\usetheme{default}
\usepackage{color}
\useoutertheme{default}
\usepackage{texnansi}
\usepackage{marvosym}
\definecolor{bottomcolour}{rgb}{0.32,0.3,0.38}
\definecolor{middlecolour}{rgb}{0.08,0.08,0.16}
\setbeamerfont{title}{size=\Huge}
\setbeamercolor{structure}{fg=gray!50!white}
\setbeamertemplate{frametitle}[default]%[center]
%\setbeamercolor{normal text}{bg=black, fg=white}
\setbeamertemplate{items}[circle]
\setbeamerfont{frametitle}{size=\Large}
\setbeamertemplate{navigation symbols}{} %no nav symbols
% Beamer theme

\useoutertheme[subsection=false]{miniframes}
\setbeamercolor{background canvas}{bg=gray!50!white}
\setbeamercolor{structure}{bg=white, fg=gray}
\setbeamercolor{frametitle}{fg=DarkChocolate, bg=gray!50}
\setbeamertemplate{itemize item}{\small\gray{$\CIRCLE$}}
\setbeamertemplate{itemize subitem}{\tiny\gray{$\CIRCLE$}}
\settowidth{\leftmargini}{\usebeamertemplate{itemize item}}
\addtolength{\leftmargini}{\labelsep}

\setbeamercovered{transparent}

\hypersetup{linkcolor=Yellow}
\hypersetup{citecolor=DeepPink4}
\hypersetup{urlcolor=DarkBlue}
\hypersetup{anchorcolor=Magenta}

\title[\hspace*{.75\linewidth}\insertframenumber/\inserttotalframenumber]
      {\fontsize{17}{17}\selectfont{\textbf{Machine learning with Python:\\[2mm]
            Create, populate \& use a \\[2mm]
            Python Virtual Environment (PVE)}}\\[6mm]
        \fontsize{13}{13}\selectfont{DuMAS Department Day}\\\medskip
        \fontsize{9}{9}\selectfont{September 22, 2023}
}
      
\subtitle{}
      
\author[{\tiny{JLC -- Sept23 -- v1.3 }}

\hspace*{.75\linewidth}]
%{\includegraphics[height=2.5cm]{images/logo-am-couleur-72dpi_alpha.jpg}\\[5mm]
{\fontsize{9}{9}\selectfont{\hspace*{-5mm}Jean-Luc.Charles@mailo.com}}

\institute{}

\date{}

\titlegraphic{\vspace*{-14mm}%\includegraphics[height=3.cm]{images/robot.png}\\
  \href{https://creativecommons.org/licenses/by-sa/4.0/}
       {\includegraphics[height=5mm]{images/CC-BY-SA.jpeg}}     
}

\logo{}

\tcbset{enhanced, boxrule=0.2pt, sharp corners, drop lifted shadow,
    width=1.0\textwidth, left=5pt, left skip=-20pt,
    colback=Chocolate!25!white,colframe=Chocolate!75!black}

\renewcommand\ttdefault{lmtt}

\begin{comment}
v1.2 -- JLC :
>>> Modification des tcolorbox pour les avoir :
- plus larges (width=...)
- mieux centrées dans la page  (left skip=...)
- avec moins de marge à gauche (left=...)
\end{comment}

\begin{document}

\frame[plain]{\titlepage}

\setbeamercolor{structure}{fg=gray!50!white}

\section{Preliminaries}
%=================================== 2 =========================================
\begin{frame}{Why use PVE to develop ML programs with Python}

  \visible<1->{
  \begin{tcolorbox}[title=PVE Advantages]
    \begin{itemize}
    \item To create a dedicated environment (disk tree) with fixed version of the Python interpreter and sensitive modules (like tensorflow)
    \item Easy to create and destroy as many times as you want
    \item To Protect your projects against operating system updates or hazardous manipulations...\\
      \visible<1->{
        $\leadsto$ \DarkGreen{you can load/update modules within a PVE without breaking\\\phantom{$\leadsto$ }modules compatibility for the other projects}
        }
    \item Each Python project should have its own PVE...
    \end{itemize}
  \end{tcolorbox}
  }
  \visible<1->{
    \bigskip
  \begin{tcolorbox}[title=Disadvantages]
    \begin{itemize}
    \item ? (just do it)
    \end{itemize}
  \end{tcolorbox}
  }

\end{frame}
%===============================================================================

\subsection{Different ways}

%=================================== 3 =========================================
\begin{frame}{Different ways to Create/Populate a PVE}
  
  \begin{tcolorbox}[title=The naive way: install the modules {\em by hand} in a {\bf conda} PVE]
    \begin{itemize}
    \item Create a PVE with \href{https://docs.conda.io/projects/conda/en/stable/}{conda} and load the desired modules {\em by hand} with 
      \codeBF{conda install module ...} or \codeBF{pip install module ...}
    \item[\unhappy] Mixing \codeBF{conda install} and \codeBF{pip install} {\em by hand} within a \codeBF{conda} PVE can lead to operating difficulties
    \item[\unhappy] Difficult to maintain, dupplicate, re-create, share...
    \end{itemize}
  \end{tcolorbox}
\vspace*{-2.5mm}
  \begin{tcolorbox}[title=The naive way: install the modules {\em by hand} in a {\bf venv}  PVE]
    \begin{itemize}
    \item Create a PVE with \href{https://docs.python.org/3/library/venv.html}{venv} and load the desired modules {\em by hand} with \codeBF{pip install module ...}
    \item[\unhappy] Most practiced by beginners, but can quickly become a mess
    \item[\unhappy] Difficult to maintain, dupplicate, re-create, share...
    \end{itemize}
  \end{tcolorbox}
\end{frame}
%===============================================================================

%=================================== 4 =========================================
\begin{frame}{Different ways to Create/Populate a PVE}
      
  \begin{tcolorbox}[title=The standard way: the {\bf venv} module \& \textbf{pip} command]
    \begin{itemize}
    \item Create the PVE with the \href{https://docs.python.org/3/library/venv.html}{venv} Python module
    \item Install the Python modules with a TXT file (list of the desired modules) and the command \codeBF{pip install -r \file{myfile.txt}}
    \item[\happy] The most used on internet tutos
    \item[\unhappy] Difficult to create a PVE with a version of Python different from the version of Python installed with your OS
    \end{itemize}
  \end{tcolorbox}

\end{frame}
%===============================================================================

%=================================== 5 =========================================
\begin{frame}{Different ways to Create/Populate a PVE}
      
  \begin{tcolorbox}[title=The preferred way: the \textbf{conda} command]
    \begin{itemize}
    \item Create the PVE with the \href{https://docs.conda.io/projects/conda/en/stable/}{conda} command 
      (available thanks to \outBF{Miniconda} distribution)
    \item Install the modules with a YAML file (list of the desired modules) and the command \codeBF{conda update ... --file \file{myfile.yml}}
    \item[\unhappy] The installation of \outBF{Miniconda} can be tricky on some computers
    \item[\happy] You can create a PVE with the version of Python you need [3.6, 3.7, 3.8 ...]
    \item[\happy] \codeBF{conda} installs transparently some modules (numpy, tensorflow...) linked with
      the \href{https://www.intel.com/content/www/us/en/docs/onemkl/get-started-guide/2023-0/overview.html}{MKL} library
      (gives high computing performances for Intel CPUs)
    \end{itemize}
  \end{tcolorbox}

\end{frame}
%===============================================================================

%=================================== 6 =========================================
\begin{frame}
  
  \begin {bclogo}[noborder=true, couleur=gray!50, couleurBarre=Chocolate, logo=\bctrombone, marge=0, margeG=-0.5]
    {\ Programming {\em Machine Learning} (ML) in Python3}
    \medskip
    \begin{itemize}
    \item \outBF{conda} or \outBF{venv} lets you install {\bf dedicated} PVEs on your laptop
      GNU/Linux, macOS or Windows for every ML project.\medskip
    \item Very often used IDE\footnote{Integrated Development Environment} for ML in Python:
      \begin{itemize}
      \item \bfdarkchoco{jupyter notebook}: for creating Python {\em notebooks} $\leadsto$ files \fileBF{*.ipynb}\\ for ML, data processing, reports... 
        Used in most tutorials on the internet.\medskip
      \item \bfdarkchoco{VSCode}, {\em a.k.a Visual Studio code} from Microsoft: multi-language, very powerful, requires some work (time) to get started,
        especially to make it work with PVE...\medskip
      \item \bfdarkchoco{idlex}: the smallest \& simplest IDE for creating/running \fileBF{*.py} files\\
        (a "Python interpreter" window and a "program editor" window)\medskip
      \item \bfdarkchoco{pycharme}, \bfdarkchoco{pyzo, }\bfdarkchoco{spyder} and many others \href{https://wiki.python.org/moin/IntegratedDevelopmentEnvironments}{here} \ldots
      \end{itemize}
    \end{itemize}
    \medskip
  \end{bclogo}
  
\end{frame}
%===============================================================================

%=================================== 7 ========================================
\begin{frame}[fragile]
%\frametitle{Installation of Python modules}

\hspace*{-5mm}Examples of YAML/TXT files to install Python modules for ML:\\[-5mm]
  \hspace*{5mm}\begin{minipage}[t]{.35\linewidth}
    \begin {bclogo}[noborder=true, couleur=gray!50, couleurBarre=Chocolate, logo=\bctrombone, marge=0, margeG=-.8]
    {\small YAML format for \codeBF{conda}}
    \begin{minted}[frame=single, fontsize=\footnotesize]{yaml}
name: dumas1
channels:
  - defaults
dependencies:
  - python=3.8
  - tensorflow==2.9.*
  - pandas
  - matplotlib
  - opencv
  - jupyter
  - notebook
  - scikit-learn
  - seaborn
  - pip
\end{minted}
    \end{bclogo}
  \end{minipage}
\hspace*{25mm}\begin{minipage}[t]{.3\linewidth}
    \begin {bclogo}[noborder=true, couleur=gray!50, couleurBarre=Chocolate, logo=\bctrombone, marge=0, margeG=-.8]
    {\small TXT format for \codeBF{pip}}
\begin{minted}[frame=single, fontsize=\footnotesize]{text}
tensorflow==2.9.*
pandas
matplotlib
opencv-python
jupyter
notebook
scikit-learn
seaborn
\end{minted}
    \end{bclogo}
    \end{minipage}  

\end{frame}
%===============================================================================

\section{Miniconda3 installation}

%=================================== 8 =========================================
\begin{frame}{}
  
  \begin{tcolorbox}[title=Installation of {\bf Miniconda3} (if not already installed)]
    \begin{itemize}
    \item Download the latest version of \outBF{Miniconda3} for your OS at \href{https://docs.conda.io/en/latest/miniconda.html}{\DarkBlue{docs.conda.io/en/latest/miniconda.html}}.\\ 
    \item Start the installation of \outBF{Miniconda3}...\\[-.3mm]
      \DarkGray{\footnotesize [Linux] In a terminal type the command: \\[-1.5mm]
        \hspace*{10pt}\code{bash ...some\_where.../miniconda3-latest-Linux-x86\_64.sh}]}
    \end{itemize}
  \end{tcolorbox}

  {\small
    \hspace*{-20pt}\textbf{Warning} : the path of the installation folder \outBF{Miniconda3} must contain no {\bf space} nor {\bf accented character}!
  }
  \begin{minipage}{.15\linewidth}
    \hspace*{-20pt}\includegraphics[width=1.\linewidth]{images/dialog-warning-2.png} 
  \end{minipage}%
  \begin{minipage}{.85\linewidth}
    
    \medskip\fontsize{6}{6}\selectfont{
    Windows :\\
    \file{C:\bsh Miconda3} ou \file{C:\bsh Users\bsh Marie\bsh miniconda3} $\leadsto$ \DarkGreen{OK}\\[1mm]
    \file{C:$\backslash$Yoann$\backslash$Mes\,\,install$\backslash$miniconda3} $\leadsto$ \DarkRed{not OK} (space)\\[1mm]
    \file{C:$\backslash$Users$\backslash$Léon$\backslash$miniconda3} $\leadsto$ \DarkRed{not OK} (accentuated e)
    
    \smallskip      
    MacOSX \& GNU/Linux :\\
    \file{/home/moi/miconda3} ou \file{/Users/moi/opt/miniconda3} $\leadsto$ \DarkGreen{OK}\\[1mm]
    \file{/Users/Léon/miniconda3} $\leadsto$ \DarkRed{not OK} (accentuated e)
    }
  \end{minipage}\medskip

  \begin{tcolorbox}[title=Already have \textbf{Anaconda3} or {\bf Miniconda3} installed on your laptop...]  
    {\footnotesize In the terminal \DarkGray{\footnotesize [macOS, Linux]}, or the "Anaconda prompt" window \DarkGray{\footnotesize [Windows]}},\\
      update \codeBF{conda} with the command:\\
    \hspace*{10pt}\codeBF{conda update -n base -c defaults conda}
  \end{tcolorbox}
\end{frame}
%===============================================================================

\subsection{post_install}

%=================================== 9 =========================================
\begin{frame}{}
  
  \begin{tcolorbox}[title=\textbf{Miniconda3} post-Installation]
    {\footnotesize In the terminal \DarkGray{\footnotesize [macOS, Linux]}, or the "Anaconda prompt" window \DarkGray{\footnotesize [Windows]}}
    \begin{itemize}
    \item if you want to disable the automatic activation of the \textbf{base} default PVE, type:\\
      \codeBF{conda config --set auto\_activate\_base false}
    \end{itemize}
  \end{tcolorbox}

  \begin{tcolorbox}[title=Info on the \textbf{conda} installation]    
    {\footnotesize In the terminal \DarkGray{\footnotesize [macOS, Linux]}, or the "Anaconda prompt" window \DarkGray{\footnotesize[Windows]}}
    \begin{itemize}
    \item to get information on the \outBF{Miniconda3} installation:\\
      \codeBF{conda info}
    \end{itemize}
  \end{tcolorbox}

  \vfill

\end{frame}
%===============================================================================

\section{conda PVE}

%=================================== 10 =========================================
\begin{frame}{Create/Populate a PVE with \textbf{conda}}

  \begin{tcolorbox}[title={\bf Create a conda PVE}]
    In a \textbf{NEW terminal} \DarkGray{\footnotesize [macOS, Linux]} or an "Anaconda prompt" window \DarkGray{\footnotesize[Windows]},
    create the \DarkRed{\boldtt{dumas1}} PVE:\\
    \hspace*{10pt}\codeBF{conda create -n dumas1 python=3.8 -y}
  \end{tcolorbox}

  \begin{tcolorbox}[title={\bf Activate a conda PVE}]
    Once the \DarkRed{\boldtt{dumas1}} PVE is created, you must {\bf activate} it to use it:
    
    \begin{itemize}
    \item In the terminal \DarkGray{\footnotesize [macOS, Linux]}, or the "Anaconda prompt" window \DarkGray{\footnotesize [Windows]}, type:\\
      \hspace*{10pt}\codeBF{conda activate dumas1}
    \item the {\em prompt} is now prefixed with \codeBF{(dumas1)}, for example:\\[-1mm]
      {\footnotesize\small
        \begin{description}
        \item{\file{Windows:\ \ \ \ }} \codeBF{(dumas1) C:\bsh Users\bsh me>}
        \item{\file{macOs:\ \ \ \ \ \ }} \codeBF{(dumas1) /Users/me \$}
        \item{\file{GNU Linux:\ \ }} \codeBF{(dumas1) user@home \$ }
        \end{description}
        }
    \end{itemize}
  \end{tcolorbox}
\end{frame}
%===============================================================================
  
%=================================== 11 ========================================
\begin{frame}{Create/Populate a PVE with \textbf{conda}}

  \hspace*{-6mm}Some useful \codeBF{conda} commands you can type in the terminal \DarkGray{\footnotesize [macOS, Linux]},\\
  \hspace*{-6mm}or the "Anaconda prompt" window \DarkGray{\footnotesize [Windows]}:
  
  \begin{tcolorbox}[title={\bf List all the conda PVEs}]
      \hspace*{10pt}\codeBF{conda env list}
  \end{tcolorbox}

  \begin{tcolorbox}[title={\bf Deactivate a conda PVE}]
    If you need to deactivate any activated PVE:\\
      \hspace*{10pt}\codeBF{conda deactivate}\\
  $\leadsto$ and the prompt is no more prefixed by any PVE name.
  \end{tcolorbox}

  \begin{tcolorbox}[title={\bf Remove a conda PVE}]
    If you want to remove the \DarkRed{\boldtt{dumas1}} PVE, type:\\
      \hspace*{10pt}\codeBF{conda deactivate}  \ttfamily{\# in case the dumas1 PVE is activated}\\
      \hspace*{10pt}\codeBF{conda env remove -n dumas1}
  \end{tcolorbox}
\end{frame}
%===============================================================================

\subsection{PVE populating}

%=================================== 12 ========================================
\begin{frame}{Create/Populate a PVE with \textbf{conda}}

  \begin{tcolorbox}[title={\bf Populate a conda PVE}]
    
    \begin{itemize}
    \item The \codeBF{--file} option of the \codeBF{conda env update} command takes the name of an ASCII file
      in YAML format containing the list of Python modules to install.
     \item It is imperative to name the selected PVE with the option: \\\codeBF{-n <PVE\_name>}
     \end{itemize}
    
\end{tcolorbox}

  \begin {bclogo}[noborder=true, couleur=gray!50, couleurBarre=Chocolate, logo=\bctrombone, marge=0, margeG=-.8]
    {Install the modules for the \codeBF{dumas1} PVE with \file{dumas1.yml}}\smallskip
    {\small In the terminal \DarkGray{\footnotesize [macOS, Linux]}, or the "Anaconda prompt" window \DarkGray{\footnotesize [Windows]}}\\[-4mm]
      \begin{itemize}
      \item with the \codeBF{cd} command go into the folder holding the YAML file:\\
       \codeBF{cd <path\_of\_the\_folder\_containing\_the\_YAML\_file>}
      \item then install the Python modules in the \codeBF{dumas1} PVE:\\
       \code{conda env update -n dumas1 --file dumas1.yml}
      \end{itemize}
  \end{bclogo}
  
  
\end{frame}
%===============================================================================

\section{venv PVE}

%=================================== 13 ========================================
\begin{frame}{Create/Populate a PVE with \textbf{venv}}

  \hspace*{-8mm} If the creation of the PVE with \codeBF{conda} fails you can switch to \codeBF{venv}:\smallskip

  
  \begin{tcolorbox}[title={\bf Create a venv PVE}]
    {\footnotesize In a terminal \DarkGray{\footnotesize [macOS, Linux]} or a "CMD" window \DarkGray{\footnotesize[Windows]}}\\
    To create a PVE, type: \hspace*{10pt}\codeBF{python -m venv <PVE\_dir>}
    \begin{itemize}
    \item \codeBF{<PVE\_dir>} is the name of the directory (or directory path) that will be created for the PVE
      (the shortest, the easiest to use...)
    \item \codeBF{<PVE\_dir>} can be as simple as \code{dumas1} or a full path like:\\
      \DarkGray{\footnotesize [Windows]}\hspace*{6mm} \code{C:\bsh Users\bsh me\bsh dir1\bsh...\bsh dumas1}\\
      \DarkGray{\footnotesize [macOS, Linux]} \code{/home/users/.../dumas1}
    \end{itemize}
    
  \end{tcolorbox}

  \begin {bclogo}[noborder=true, couleur=gray!50, couleurBarre=Chocolate, logo=\bctrombone, marge=0, margeG=-.8]
    {Create the \codeBF{dumas1} PVE}\smallskip
    In a terminal \DarkGray{\footnotesize [macOS, Linux]} or a "CMD" window \DarkGray{\footnotesize[Windows]}:\\
      \hspace*{10pt}\codeBF{python -m venv dumas1}
  \end{bclogo}
  
\end{frame}
%===============================================================================

%=================================== 14 ========================================
\begin{frame}{Create/Populate a PVE with \textbf{venv}}
  
  \begin{tcolorbox}[title={\bf Activate a venv PVE}]
    Once the \codeBF{dumas1} PVE is created, you must {\bf activate} it to use it:
    
    \begin{itemize}
    \item \DarkGray{\footnotesize [Windows]} \codeBF{<PVE\_dir>\bsh Scripts\bsh activate.bat}
    \item \DarkGray{\footnotesize [macOS, Linux]} \codeBF{source <PVE\_dir>/Scripts/activate}
    \item the {\em prompt} is prefixed with the activated PVE:\\[-1mm]
      {\footnotesize
        \begin{description}
        \item{\file{Windows:\ \ \ \ }} \codeBF{(dumas1) C:\bsh Users\bsh me>}
        \item{\file{macOs:\ \ \ \ \ \ }} \codeBF{(dumas1) /Users/me \$}
        \item{\file{GNU Linux:\ \ }} \codeBF{(dumas1) user@home \$}
        \end{description}
      }
    \end{itemize}
  \end{tcolorbox}

  \begin {bclogo}[noborder=true, couleur=gray!50, couleurBarre=Chocolate, logo=\bctrombone, marge=0, margeG=-.8]
    {Activate the \codeBF{dumas1} PVE}\smallskip
    {\small In the "CMD" window \DarkGray{\footnotesize[Windows]}}:\\
    \hspace*{10pt}\codeBF{dumas1\bsh Scripts\bsh activate.bat}
    
    {\small In the terminal \DarkGray{\footnotesize [macOS, Linux]}}:\\
    \hspace*{10pt}\codeBF{source dumas1/Scripts/activate}

    Check the {\em prompt}: it should be be prefixed by \codeBF{dumas1}...

  \end{bclogo}
\end{frame}
%===============================================================================

%=================================== 15 ========================================
\begin{frame}{Create/Populate a PVE with \textbf{conda}}

  \begin{tcolorbox}[title={\bf Populate a venv PVE}]  
    \begin{itemize}
    \item The \codeBF{-r} option of the \codeBF{pip install} command takes the name of a TXT file
      containing the list of the Python modules to install.
    \item beforehand, the {\bf target PVE must have been activated}
    \end{itemize}
  \end{tcolorbox}

  \begin {bclogo}[noborder=true, couleur=gray!50, couleurBarre=Chocolate, logo=\bctrombone, marge=0, margeG=-.8]
    {Install the modules in the \DarkRed{\boldtt{dumas1}} PVE with \file{dumas1.txt}}
    {\small In a terminal \DarkGray{\footnotesize [macOS, Linux]} or a "CMD" window \DarkGray{\footnotesize[Windows]}}
    with the \codeBF{dumas1} PVE activated:\\
    \hspace*{10pt}\codeBF{pip install -r \file{path\_of\_dumas1.txt} -y}
  \end{bclogo}

  \vfill
  
\end{frame}
%===============================================================================

\section{Launch Jupyter notebook}

%=================================== 16 ========================================
\begin{frame}

  \begin{tcolorbox}[title={Launch {\bf jupyter notebook}}]
 
    In a terminal \DarkGray{\footnotesize [macOS, Linux]} or a "CMD/Anaconda prompt" window \DarkGray{\footnotesize[Windows]},
      with the \codeBF{dumas1} PVE {\bf activated}, type:\\
      \hspace*{10pt}\codeBF{jupyter notebook}
  \end{tcolorbox}
  
  \begin{tcolorbox}[title={Access notebooks anywhere with {\bf jupyter}}]
    \begin{itemize}
    \item In a "CMD/Anaconda prompt" window \DarkGray{\footnotesize[Windows]}, access notebooks on a logical unit other than \file{C:\bsh}
      (partition disk \file{D:\bsh}, USB key \file{E:\bsh}):\\
      \hspace*{10pt}\codeBF{jupyter notebook D:\bsh}\\
      \hspace*{10pt}\codeBF{jupyter notebook D:\textbackslash{folder1}\texttt{\char`\\}folder2}\\
    \item In a terminal \DarkGray{\footnotesize [macOS, Linux]}, access the notebooks in a given folder:\\
      \hspace*{10pt}\codeBF{jupyter notebook /home/users/me/folder1/folder2}
    \end{itemize}
  \end{tcolorbox}

  \begin{tcolorbox}[title={Quit {\bf jupyter notebook}}]      
    In a terminal \DarkGray{\footnotesize [macOS, Linux]} or a "CMD" window \DarkGray{\footnotesize[Windows]},
    after quitting the jupyter window, type CTRL-C twice to interrupt the jupyter server.
  \end{tcolorbox}  
\end{frame}
%===============================================================================

\section{Launch idlex}

\subsection{Windows}

\end{document}









\subsection{with pip and a requirement file}

%===============================================================================
\begin{frame}{Installation des modules Python}
  
  \begin{tcolorbox}[title=Fichier ASCII + commande \textbf{pip}]
    \begin{itemize}
    \item L'option \codeBF{-r} de la commande \codeBF{pip} prend le nom d'un fichier ASCII contenant la liste des modules à installer.
    \item Il faut impérativement désigner l'PVE concerné avec l'option
      
      \codeBF{-n <PVE\_name>} 
    \end{itemize}
  \end{tcolorbox}

  \begin {bclogo}[noborder=true, couleur=gray!50, couleurBarre=Chocolate, logo=\bctrombone, marge=0, margeG=-.8]
    {Installer l'PVE bam avec un fichier \file{requirements.txt}}\smallskip
    Dans un terminal (fenêtre "Anaconda prompt") avec l'PVE {\bf bam activé}, taper:\\
    \Chocolate{\boldtt{pip install -r <chemin\_du\_fichier\_requirements.txt}}\\
  \end{bclogo}
  
\end{frame}
%===============================================================================


\subsection{by hand}

%===============================================================================
\begin{frame}{Installation of Python module}
  
  \begin{tcolorbox}[title={\bf for information only }: By hand...]
      \begin{itemize}
      \item With \codeBF{conda}:\\
        \codeBF{conda install module1 modle2 ...}\\
        \codeBF{conda install module1==x.y.z ...} {\small (force version)}\\[1mm]
      \item With \codeBF{pip}:\\
        \codeBF{pip install module1 modle2 ...}\\
        \codeBF{pip install module1==x.y.z ...} {\small (force version)}
      \end{itemize}
  \end{tcolorbox}

  \begin {bclogo}[noborder=true, couleur=gray!50, couleurBarre=Chocolate, logo=\bctrombone, marge=0, margeG=-.8]
    {Install bam PVE modules manually}
    {\small \DarkRed{$\leadsto$ FYI, not recommended: better to use "YAML file + \codeBF{conda}..."}
       
     In a terminal, or an "Anaconda prompt" window, with the bam PVE {\bf activated}, type successively:\\
     \Chocolate{\boldtt{conda install numpy scipy sympy pandas matplotlib imageio opencv}}\\
     \Chocolate{\boldtt{conda install jupyter notebook jupyterlab}}\\
     \Chocolate{\boldtt{pip install idlex}}
     }
  \end{bclogo}
  
\end{frame}
%===============================================================================

%===============================================================================
\begin{frame}

\begin{tcolorbox}[title=\textbf{Windows}: Shortcut to launch \textbf{Idlex}]

    {\small
    \file{<MC>} designating the path to the folder \file{miniconda3} on your computer:
     \begin{itemize}
     \item In the folder \fileBF{<MC>\bsh envs\bsh dumas1\bsh Scripts}: <<\;right-click\;>> on the file \fileBF{idlex.py} to create a <<\; desktop shortcut\;>>.
     \item In the desktop: <<\;right-click\;>> on the \fileBF{idlex.py} shortcut to modify its <<\;properties\;>> in the <<\;Shortcut\; >> tab:\smallskip
     \begin{itemize}
         \item field <<\;target\;>>: put {\bf on a single line}:\\
           \hspace*{-10mm}\code{\fontsize{6}{6}\selectfont{<MC>\bsh condabin\bsh conda.bat activate dumas1 \& <MC>\bsh envs\bsh dumas1\bsh python.exe <MC>\bsh envs\bsh dumas1\bsh Scripts\bsh idlex.py}}\\
           \hspace*{-8mm}\DarkGreen{\footnotesize $\leadsto$ replace <MC> with the path of the folder {\em miniconda3} on your computer.}
           \smallskip
         \item field <<\;Start in\;>>: put the path of a folder in your user tree, or the path of your user folder.
           \smallskip
         \item field <<\;Icon\;>>: click on <<\;Change icon $\leadsto$ Browse\;>> and install the icon \file{<MC>\bsh Lib\bsh idlelib\ bsh Icons\bsh idle.ico}
     \end{itemize}
   \end{itemize}
   }
  \end{tcolorbox}
 
   \begin{tcolorbox}[title={\bf macOS \& GNU/Linux}: launch \textbf{Idlex}]
   In a terminal with the \DarkRed{\boldtt{dumas1}} PVE {\bf activated}, type: \codeBF{idlex}
   \end{tcolorbox}
   
\end{frame}
%===============================================================================




\subsection{3 ways}

%===============================================================================
\begin{frame}{Installation of Python modules}
  
  \begin {bclogo}[noborder=true, couleur=gray!50, couleurBarre=Chocolate, logo=\bctrombone, marge=0, margeG=-.8]
    {Different methods to load modules into the \DarkRed{\boldtt{dumas1}} PVE}\medskip
    
     \begin{itemize}
     \item With an ASCII-YAML file (\file{*.yml}) listing the modules to install and
       the command \codeBF{conda env update -n dumas1 --file <file.yml>}:\\[1mm]
       $\leadsto$ the most efficient for a PVE created with \codeBF{conda}\medskip
      
     \item With an ASCII file (\file{*.txt}) listing the modules to install and
       the command \codeBF{pip install -r <file.txt>}:\\
       $\leadsto$ the most used on internet tutorials... \\
       $\leadsto$ but can lead to incompatibilities between \codeBF{conda} and \codeBF{pip}\medskip
      
     \item By hand with \codeBF{conda install ...} or \codeBF{pip install ...}:\\
       $\leadsto$ the most painful!
     \end{itemize}    
  \end{bclogo}

\end{frame}
%===============================================================================



